Um distribuidor de carga pode ser usado em bases de dados para oferecer alta disponibilidade e ganho de performance em pedidos de leitura apenas. Este projecto visa analisar a escalabilidade de uma base de dados PostgreSQL com o PGpool-II como distribuidor de carga. Como ferramenta de \textit{benchmark} utilizou-se a ferramenta PGbench. Observando os resultados obtidos é possível concluir que a instalação do PGpool num \textit{cluster} de bases de dados só será benéfica caso este seja alvo, maioritariamente, de transacções de leitura, caso contrário deverá ser dada atenção ao deterioramento da performance das transacções de escrita, tentando procurar soluções para mitigar este problema.

\keywords{Postgres  \and Scalability \and Pgpool \and Benchmark \and Pgbench.}