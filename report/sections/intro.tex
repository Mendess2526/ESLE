\subsection{Introdução}
O \textit{paper} escolhido~\cite{paper} explora a comparação de duas implementações distintas de distribuidores de carga para dois sistemas de base de dados conhecidos, MySQL e PostgreSQL. Porém, como o objectivo deste projecto era apenas analisar a escalabilidade de um sistema que respeitasse um conjunto de requisitos indicados num curto espaço de tempo e dado que o Postgres é um sistema robusto em produção, decidiu-se então focar a análise apenas no caso do PostgreSQL. 

A escolha do \textit{paper} deve-se à constante necessidade de sistemas de base de dados, nos dias de hoje, e como tal de que estes se mantenham eficientes e rápidos com o aumento da carga, fazendo com que, neste caso, a replicação e os distribuidores de carga ajudem nesse sentido. Assim, é necessário verificar a escalabilidade de tais sistemas ao introduzir réplicas e/ou distribuidores de carga.

\subsection{Âmbito}

Este projecto \footnote{Repositório Git - https://github.com/Mendess2526/ESLE-Postgres/releases/tag/1.0} foi desenvolvido no âmbito da unidade curricular de Engenharia de Sistemas de Larga Escala do Instituto Superior Técnico (Taguspark).

\subsection{Objectivo}

O projecto teve como objectivo analisar a escalabilidade de uma base de dados PostgreSQL ao adicionar réplicas da mesma e o distribuidor de carga PGpool-II. Para além disso, destinou-se também à utilização dos conceitos apresentados nas aulas teóricas da unidade curricular.